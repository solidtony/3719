%20130916 JW, first writing
\documentclass[iop]{emulateapj}

\def\g{$g$}
\def\fillin{\#\#\#\#????}

\begin{document}

\title{Humpty Dumpty Had a Great Fall: Measuring the Acceleration due to Gravity with a Simple Pendulum}

\author{ Anthony Garcia
and
Joshua Wallace}
\affil{Advanced Undegrad Lab,Department of Physics and Astronomy, University of Utah,
 115 South 1400 East, Salt Lake City, UT 84112
}

\begin{abstract}

\g man, \g!

\end{abstract}

\section{Introduction}

Of the four fundamental forces of nature, gravity is the one that we have the 
most experience with in our everyday lives.  It is such a constant in our lives 
that every movement we make factors in the force of gravity, even if we are not
aware of it. We put objects down, expecting them to stay there because of 
gravity. A basketball player shoots a ball upwards, expecting the familiar 
parabolic arc due to gravity to bring it back down into the basket.  Even the
common phenomenon known as walking relies on gravity's pull to give our feet
the friction needed to push us forward. Our experience with gravity is so
intimate that we know exactly how things perform in earth's gravity field 
without having to measure and calculate (e.g. the basketball player knowing the 
right momentum to give the ball to score a basket without any calculation being
performed).

\g as a measure of force of gravity...

Importance of calculating g:  study mantle, earthquakes?, earth's density 
(similar to recent lunar probes), clock timing (which use pendula as well)...

To calculate \g, we will use a pendulum setup quite similar to that of the
pendulum clocks mentioned above.  Section~\ref{sec:theory} will introduce the 
physics behind this calculation more thoroughly.  Here we state the intended 
goals of our study of the motion of a simple pendulum:  (1) To be able to 
determine the value of \g, which as discussed above is an important parameter
for a variety of calculations; (2) to test a mathematical model (introduced in 
section~\ref{sec:theory}) of a physical system and compare the correspondence 
between the two and (3) use the process as an introduction into the sources of 
errors and their propagation to gain greater skills as a physicist-in-training. 
The experiment we will be performing is discussed in great detail in .......... 

\section{Theoretical Background}
\label{sec:theory}


Later gater...








\section{Experimental Procedure}
\label{sec:procedure}

\subsection{Setup}

The setup consisted of a metal frame attached to a wall, a string, and a metal 
ball.

\subsection{Procedure}

The first measurements we made were those of the ball diameter and the length 
of the hook on the top of the ball.  These data are contained in table \#\#\#\#\#. 
These measurements were made using ???\#\#\#a vernier caliper.  Due to the size of 
the ball and the large uncertainty associated with measuring the full length 
from the pivot point to the ball center (the center of mass of the
ball) we decided it would be best to measure the length of the string from the
pivot point to the top of the hook of the ball and then add the measured values 
for the hook height and ball radius afterwards.  The uncertainties from the 
caliper measurements are small enough compared to the uncertainty in using 
either a measuring stick or a tape measure in determining the length of the
string that making three separate measurements like this do not create an 
unnecessarily large uncertainty in length.  As stated above, a tape measure 
and/or measuring stick were used to measure the distance from the pivot point
to the top of the hook.  This was the distance that was measured when the
pendulum length was varied.

The next step was to determine the uncertainty of the stopwatch we used.  The
stopwatches we used were brand \#\#\#\#\#.  There were two sources of systematic 
uncertainty in the time measurements made by the stopwatch.  The first was the 
gain/lag the stopwatch had over real time.  The second was simply the finite 
gradations of time measured by the stopwatch ($.01 s$).  To account for the 
first source of systematic error (and also to determine which of the two 
authors were most accurate and precise in their time measurements) we timed 
ourselves against an atomic clock for 1s and 100s intervals.  (On the second
day, 20s and 50s intervals).  These data are available in table \fillin.  JW 
was determined to have the best combination of accuracy and precision and was 
designated as the official timekeeper for the duration of the study.  As seen 
in table \fillin, the mean of the time measurements were not equal to the time 
interval attempted to be measured.  This gain/lag were accounted for in our 
final results by finding an appropriate ``weighting factor'' by which to 
multiply all our time measurements to increase their accuracy.  The weighting 
factor was calculated as follows:

\begin{equation}
\label{eq:timeweight}
W = 1 + \frac{t_{atomic}-t_{mean}}{t_{atomic}} 
\end{equation}

where $t_{atomic}$ is the time according to the atomic clock (also, the time we 
were trying to measure) and $t_{mean}$ is the mean of the times measured by the 
stopwatch.







\section{Data and Results}
\label{sec:data}













\section{Discussion}
\label{sec:discuss}
















\section{Conclusion}














\acknowledgments

We thank ....





\begin{thebibliography}{}

%\bibitem[

\end{thebibliography}{}

\end{document}
