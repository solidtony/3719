%20130916 JW, first writing
\documentclass[iop]{emulateapj}

\def\g{$g$}

\begin{document}

\title{Humpty Dumpty Had a Great Fall: Measuring the Acceleration due to Gravity with a Simple Pendulum}

\author{ Anthony Garcia
and
Joshua Wallace}
\affil{Advanced Undegrad Lab,Department of Physics and Astronomy, University of Utah,
 115 South 1400 East, Salt Lake City, UT 84112
}

\begin{abstract}

\g man, \g!

\end{abstract}

\section{Introduction}

Of the four fundamental forces of nature, gravity is the one that we have the 
most experience with in our everyday lives.  It is such a constant in our lives 
that every movement we make factors in the force of gravity, even if we are not
aware of it. We put objects down, expecting them to stay there because of 
gravity. A basketball player shoots a ball upwards, expecting the familiar 
parabolic arc due to gravity to bring it back down into the basket.  Even the
common phenomenon known as walking relies on gravity's pull to give our feet
the friction needed to push us forward. Our experience with gravity is so
intimate that we know exactly how things perform in earth's gravity field 
without having to measure and calculate (e.g. the basketball player knowing the 
right momentum to give the ball to score a basket without any calculation being
performed).

\g as a measure of force of gravity...

Importance of calculating g:  study mantle, earthquakes?, earth's density 
(similar to recent lunar probes), clock timing (which use pendula as well)...

To calculate \g, we will use a pendulum setup quite similar to that of the
pendulum clocks mentioned above.  Section~\ref{sec:theory} will introduce the 
physics behind this calculation more thoroughly.  Here we state the intended 
goals of our study of the motion of a simple pendulum:  (1) To be able to 
determine the value of \g, which as discussed above is an important parameter
for a variety of calculations; (2) to test a mathematical model (introduced in 
section~\ref{sec:theory}) of a physical system and compare the correspondence 
between the two and (3) use the process as an introduction into the sources of 
errors and their propagation to gain greater skills as a physicist-in-training. 
The experiment we will be performing is discussed in great detail in 









\section{Theoretical Background}
\label{sec:theory}











\section{Experimental Procedure}
\label{sec:procedure}

\subsection{Setup}



\subsection{Procedure}











\section{Data and Results}
\label{sec:data}













\section{Discussion}
\label{sec:discuss}
















\section{Conclusion}














\acknowledgments

We thank ....





\begin{thebibliography}{}

%\bibitem[

\end{thebibliography}{}

\end{document}
