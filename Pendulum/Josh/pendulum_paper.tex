%20130916 JW, first writing
\documentclass[iop,numberedappendix,appendixfloats]{emulateapj}

\def\g{$g$}
\def\t{$t$}
\def\T{$T$}
\def\Ttwen{$T_{20}$}
\def\fillin{\#\#\#\#????}
\def\deg{^{\circ}}

\begin{document}

\title{Humpty Dumpty Had a Great Fall: Measuring the Acceleration due to Gravity with a Simple Pendulum}

\author{ Anthony D. Garcia
and
Joshua J. Wallace}
\affil{Advanced Undegraduate Lab,Department of Physics and Astronomy, University of Utah,
 115 South 1400 East, Salt Lake City, UT 84112
}

\begin{abstract}

\g man, \g!

\end{abstract}

\section{Introduction}

Of the four fundamental forces of nature, gravity is the one that we have the 
most experience with in our everyday lives.  It is such a constant in our lives 
that every movement we make factors in the force of gravity, even if we are not
aware of it. We put objects down, expecting them to stay there because of 
gravity. A basketball player shoots a ball upwards, expecting the familiar 
parabolic arc due to gravity to bring it back down into the basket.  Even the
common phenomenon known as walking relies on gravity's pull to give our feet
the friction needed to push us forward. Our experience with gravity is so
intimate that we know exactly how things perform in earth's gravity field 
without having to measure and calculate (e.g. the basketball player knowing the 
right momentum to give the ball to score a basket without any calculation being
performed).

\g\ as a measure of force of gravity...

Importance of calculating g:  study mantle, earthquakes?, earth's density 
(similar to recent lunar probes), clock timing (which use pendula as well)...

To calculate \g, we will use a pendulum setup quite similar to that of the
pendulum clocks mentioned above.  Section~\ref{sec:theory} will introduce the 
physics behind this calculation more thoroughly.  Here we state the intended 
goals of our study of the motion of a simple pendulum:  (1) To be able to 
determine the value of \g, which as discussed above is an important parameter
for a variety of calculations; (2) to test a mathematical model (introduced in 
section~\ref{sec:theory}) of a physical system and compare the correspondence 
between the two and (3) use the process as an introduction into the sources of 
errors and their propagation to gain greater skills as a physicist-in-training. 
The experiment we will be performing is discussed in great detail in .......... 

\section{Theoretical Background}
\label{sec:theory}


Later gater...

Also talk about small angle approximation...






\section{Experimental Procedure}
\label{sec:procedure}

\subsection{Setup}

The setup consisted of a metal frame attached to a wall, a string, and a metal 
ball.

\subsection{Procedure}

The first measurements we made were those of the ball diameter and the length 
of the hook on the top of the ball.  Due to the size of 
the ball and the large uncertainty associated with measuring the full length 
from the pivot point to the ball center (the center of mass of the
ball) we decided it would be best to measure the length of the string from the
pivot point to the top of the hook of the ball and then add the measured values 
for the hook height and ball radius afterwards.  The uncertainties from the 
caliper measurements are small enough compared to the uncertainty in using 
either a measuring stick or a tape measure in determining the length of the
string that making three separate measurements like this do not create an 
unnecessarily large uncertainty in length.  As stated above, a tape measure 
and/or measuring stick were used to measure the distance from the pivot point
to the top of the hook.  This was the distance that was measured when the
pendulum length was varied.

The next step was to determine the uncertainty of the stopwatch we used.  The
stopwatches we used were brand \#\#\#\#\#.  There were two sources of systematic 
uncertainty in the time measurements made by the stopwatch.  The first was the 
gain/lag the stopwatch had over real time.  The second was simply the finite 
gradations of time measured by the stopwatch ($.01 s$).  To account for the 
first source of systematic error (and also to determine which of the two 
authors were most accurate in their time measurements) we timed 
ourselves against an atomic clock for 1s and 100s intervals.  (On the second
day, 20s and 50s intervals). JW 
was determined to have the best accuracy and was 
designated as the official timekeeper for the duration of the study.  
We then used the following equation to find how much the stopwatch time
differed from the atomic clock time (defined as $S$):

\begin{equation}
\label{eq:S}
S=\frac{t_{atomic}-t_{mean}}{t_{atomic}} 
\end{equation}

where $t_{atomic}$ is the real time as measured by the 
atomic clock 
and $t_{mean}$ is the mean of the times measured by the 
stopwatch.  This ratio was multiplied by all the \T\ we measured to give
the error from the gain/lag of the stopwatch.

Measuring against the atomic clock provided a measure of both the stopwatch's 
own limitations and the user's reaction time.  However, it is impossible to 
separate the two.  In order to fully account for the user's reaction time the 
above error was assumed to be due entirely to the stopwatch's limitations and 
a separate calculation was made of the timekeeper's reaction time.  This was 
done by having the timekeeper push the start and stop button on the 
stopwatch as fast as possible.  The average of these times was included in 
the overall error in measuring \T.

The next thing we did was determine the best way to measure the period of the 
pendulum's swing, i.e., a method that would allow us to both be accurate 
(allowing us to obtain a true value for \g) and precise (thus reducing random 
error). We first compared whether it was better to start timing a period from 
when the ball was released or to wait a period and start timing with the 
second period.  We wanted to see if there was a difference in accuracy and 
precision between the simultaneous double hand movement of releasing the ball 
while starting the stopwatch and just timing the period by eye. For this we 
timed a single swing of the pendulum eight times for both cases (i.e. starting 
timing with the swing and starting timing with the second swing). $\theta_{max}$ 
{\bf check thetamax!} was kept small (i.e., about $5\deg$) to ensure the 
validity of the small angle approximation (a more detailed analysis of what 
consitutes a "small angle" is provided in section \fillin). This was done 
for two different lengths: one around $26cm$ and the other around \fillin$cm$. 
$\theta_max$ {\bf check thetamax!} was kept small (i.e., about $5\deg$) 

Unfortunately the method with the lowest 
mean (timing from release) had the largest spread in times measured and 
vice-versa (for timing the second period). We chose to go 
with the method with the least uncertainty (i.e. lowest spread, or timing the 
second swing) over greatest accuracy because we figured that the method we 
used to determine the stopwatch gain/lag was closest to the method that timed 
the second swing; for both methods, the start and stop time were determined by 
eye.  If we had pushed a button on the atomic clock to start its timing at the 
same moment we had started the stopwatch, then this method would have been 
more similar to the method that used the release time of the pendulum to 
measure \T.  Therefore, the ratio $S$ determined by equation~\ref{eq:S} 
factors in the inaccuracy of the second period-measuring method more 
than the first method (in addition to the stopwatch's own gain/lag). The 
inaccuracy of the second method being thus accounted for we are free to choose 
the method that provides the lowest random uncertainty. 

We then measured the period of 20 oscillations (starting from the second 
swing) five separate times ($\theta_{max}$ was again kept around $5\deg$). 
We refer to this measurement as $T_{20}$. The purpose of this 
measurement was to reduce the error in determining the period.  Since 
$T=\frac{T_{20}}{20}$ it follows that
\begin{equation}
\label{eq:T20error}
\delta_T = \frac{\delta_{T20}}{20}.  Derive this later
\end{equation}
A more detailed derivation of equation~\ref{eq:T20error} is available in 
Appendix B ({\bf Provide more specific equation number in appendix!!}).
Thus, measuring 20 periods allows us to reduce the error in \T\ by a factor of 
20.  Since measuring 20 periods is not more difficult or significantly more 
time consuming than just measuring one period, this provides a relatively easy 
way to reduce the uncertainty in \T.  ?After measuring \Ttwen\ five times we 
found the mean, the standard deviation, ???and the standard deviation of the 
mean?.  As will be shown in section~\ref{sec:data}, five was a good number of trials to 
choose to reduce the random uncertainty to values comparable to the systematic 
uncertainties without having to perform the experiment and unnecessarily large 
number of times. {\bf We must show this!!!!}  The length of the string was 
then increased to the previous mentioned second length of 
about \fillin $cm$ and the above process was repeated (i.e., 
timing the first period, timing the second period, timing 20 periods).

A value for \g\ was then calculated for each of the above measurements (6 in 
total). Each was compared to the accepted value of \g, \fillin.  {\bf The 
method with the least uncertainty in \g\ was taken to be our method of choice 
in determining a more accurate value of \g\ later.  The method chosen was the 
"\Ttwen\" method, where the time of 20 periods are measured and the value of 
\T\ is found by dividing \Ttwen\ by 20.  As shown in 
equation~\ref{eq:T20error} the error of \Ttwen\ is also divided by 20, thus 
causing this method to lead to greater precision.  {\bf hopefully it will lead 
to greater accuracy as well.  We must find out}. 

To minimize the uncertainty in measuring the string length, we used two 
different measuring methods to see which we felt gave us the most accuracy. 
We used a measuring stick and a tape measure.  We felt the tape measure gave 
us the most accurate measurement because we could put the $0cm$ end right at 
the pivot point (as opposed to the measuring stick, which required us to line 
up the stick with the pivot point level, adding in uncertainty, and also 
required us to reset it every time we wanted to make a measurement since it 
was in the way of the ball swing. The tape measure had this same situation 
as well, but since we could place its end directly on the pivot point level 
we knew we would have the same zero point every time. Thus, any error in 
measuring the zero point with the tape measure would be the same for all 
length measurements made by the measuring stick.  Since \g\ will be found 
from the slope of equation~\fillin, an error that is the same for all length 
measurements will not affect the calculated value of \g.  The measuring stick, 
on the other hand, would have to have its zero point reset for every length 
measure, thus providing random error in the length measurements that would 
not be the same for every length measurement and could affect the calculated 
value of \g).  It is important to note that the measuring tape's metal claw 
moves depending on whether you are measuring inside or outside and object. 
Since we were measuring inside an object (i.e., underneath the metal frame 
where the pivot point was located) we made sure that the metal claw was always 
pushed down towards the measuring tape to give us an accurate measure.  
Another reason the tape measure was chosen over the measuring stick was 
because it allowed us a more accurate measure at the ball's end of the string. 
Since the ball had a finite size, it got in the way of a direct measure of the 
top of the hook's location ("direct" meaning being able to lay the hook on the 
measuring stick and seeing which line it matched). {\bf include?:  We also could not rotate 
the ball out of the way of the measuring stick, since doing so required us to 
take a part of the ball's weight into our hand.  The problem with this was 
that the string we used was not truly inextensible as is ideal and the weight 
of the ball had an effect on the length of the string.  Taking the weight of 
the ball off the string and into our hand caused the string to relax from its 
fully extended length (how accurate can we determine this?}  The measuring 
tape, on the other hand, provided the flexibility necessary to bend the 
measuring device around the ball, allowing us to get a direct measure of the 
length (great care was taken to ensure that portion of the tape measure that 
was bent was past the portion of the tape measure involved in measuring the 
length so that the measurement was as accurate as possible). For long lengths 
of the string it became necessary for two people to operate the tape measure:
one to read the measurement at the bottom of the tape measure and one to hold 
claw next to the pivot point.  Care was taken that the claw was pushed down 
towards the measuring tape as described above.  Statisitical information was 
also obtained on the length of the pendulum by having each measured twice
......%We didn't measure everthing twice, and Yanil said we didn't need 
%to, I think so maybe shouldn't include this part.

Parallax...
%Still don't know how to use it...

We next tested the limits of validity for the small angle approximation and 
the accuracy of the model expressed by equation~\fillin. %large angle model
For three different lengths we measured \Ttwen\ for $\theta_{max} = 3\deg, 
5\deg, 10\deg, 20\deg,$ and $40\deg$.  %From these measurements it would seem 
%that $\theta_{max}\leq 10\deg$
After verifying the model of equation~\fillin\ we calculated that 
%This fillin corresponds to the model given in the handout
$\theta_{max}=5\deg$ will give us a model uncertainty small enough compared 
to our systematic uncertainties to be negligible.  See Appendix B section 
\fillin\ for a more detailed analysis.
%This section will be our calculation of T_err less than .1 percent
%should get theta_max=7.2.  5 degrees is within our error of 7.2.

Having established all our systematic uncertainties and determining the best 
method to measure \T for the resources we had, we then turned to making a more 
accurate measurement of \g.  To do this, we chose ten lengths in range from 
$~.3m$ to $~1.5m$.  Since we had recently discovered an easier way to use the 
measuring stick to make length measurements, each length was measured with 
both the tape measure and the measuring stick.  Since the measuring stick 
still proved hard to use to make accurate measurements and to keep continuity 
with our previous length measurements, we decided to use the lengths measured 
with the tape measure in our final calculation of \g. We then measured 
\Ttwen\ five times for each length.  

One thing we learned in measuring \T\ (likewise \T20) was how to accurately 
determine what constituted the beginning and end of a period.  The best 
method we found was to concentrate on the classical turning points. {\bf 
Should we include this?  We don't have data to back this up.}

Anything else to discuss?  Pitfalls?










\section{Data and Results}
\label{sec:data}













\section{Discussion}
\label{sec:discuss}

Coriolis force, moment of inertia, air friction...














\section{Conclusion}














\acknowledgments

We thank Lia Papadopolous and Zach Carson for helpful conversations and 
suggestions.






\begin{thebibliography}{}

%\bibitem[

\end{thebibliography}{}

\end{document}
